\chapter{DOS DIREITOS E OBRIGAÇÕES DOS PASSAGEIROS}

\begin{enumerate}[resume, label=Art. \arabic*]

\item São direitos dos passageiros, além daqueles previstos em legislação específica:

\begin{enumerate}[label=\roman*.]

\item receber serviço adequado e ser transportado com pontualidade, em condições de higiene, conforto e segurança, durante toda viagem;

\item ser atendido com presteza e urbanidade pelo preposto da empresa Delegatária, pela fiscalização do DER-MG e pelo pessoal credenciado ou autorizado;

\item ter garantido o seu assento no veículo, nas condições especificadas no bilhete de passagem;

\item ter transportada gratuitamente a sua bagagem, nos termos deste Regulamento;

\item ser indenizado pelo extravio ou perda da bagagem etiquetada, nos termos deste Regulamento;

\item registrar reclamação, sugestão ou elogio ao serviço, por meio do número de telefone e ou do endereço eletrônico, fixados nos veículos dos Sistemas Intermunicipal e Metropolitano de Passageiros, ou recorrer ao agente fiscal do DER-MG;

\item ser auxiliado no embarque e desembarque;

\item ter assegurada a continuidade do transporte, quando, em conseqüência de problemas no veículo ou tripulação, ocorrer interrupção de viagens;

\item ter assegurada alimentação e hospedagem na impossibilidade de continuação da viagem;

\item receber, em caso de acidente, imediata e adequada assistência por parte da Delegatária; e

\item ser reembolsado do valor do bilhete de passagem não utilizado, nos termos deste Regulamento.

\end{enumerate}

\item São obrigações dos passageiros, além daquelas previstas em legislação específica:

\begin{enumerate}[label=\roman*.]

\item pagar o preço de passagem e tarifa de embarque cobrada pela empresa Delegatária, autorizados pela SETOP;

\item portar o bilhete de passagem e documento de identificação;

\item comunicar ao DER-MG irregularidades ou atos ilícitos na prestação do serviço;

\item manter desimpedido o corredor do veículo;

\item preservar os bens vinculados à prestação do serviço;

\item zelar pela conservação e higiene do veículo;

\item tratar com urbanidade os prepostos da Delegatária, os agentes fiscais do DER-MG e os demais passageiros;

\item guardar, zelar e responsabilizar pela sua bagagem não etiquetada no porta embrulhos;

\item respeitar os tempos previstos nos pontos de parada;

\item conferir os dados constantes no seu bilhete de passagem;

\item usar somente o assento com o número constante em seu bilhete de passagem;

\item permanecer com seus aparelhos celulares desligados ou no modo silencioso ou vibratório;

\item não comprometer a segurança, o conforto e a tranqüilidade dos demais passageiros;

\item não perturbar a tripulação e os demais passageiros durante a viagem;

\item não conversar com o motorista, a não ser em caso emergencial;

\item não fumar no interior do veículo;

\item não viajar em estado de embriaguez ou sob efeito de substâncias tóxicas de qualquer natureza;

\item não fazer uso de aparelho sonoro durante a viagem, exceto se usado com fones de ouvido e sem incomodar os demais; e

\item não viajar nos degraus das escadas em frente às portas do veículo.

\end{enumerate}

\end{enumerate}