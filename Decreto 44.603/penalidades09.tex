\chapter{DAS PENALIDADES}

\begin{enumerate}[resume, label=Art. \arabic*]

\item Pela não observância do presente Regulamento, a Administração poderá, garantida a prévia defesa, aplicar às Delegatárias dos Sistemas de Transporte Coletivo Intermunicipal e Metropolitano de Passageiros as seguintes penalidades:

\begin{enumerate}[label=\roman*.]

\item multa, na forma prevista neste Regulamento;

\item advertência escrita;

\item suspensão temporária de participação em licitação e impedimento de contratar com a Administração por prazo não superior a dois anos;

\item declaração de inidoneidade para licitar ou contratar com a Administração Pública enquanto perdurarem os motivos determinantes da punição ou até que seja promovida a reabilitação perante a própria autoridade que aplicou a penalidade, que será concedida sempre que o contratado ressarcir a Administração pelos prejuízos resultantes e após decorrido o prazo da sanção aplicada com base no inciso III.

\end{enumerate}

Parágrafo único. As sanções previstas nos incisos II, III e IV poderão ser aplicadas juntamente com a de multa, assegurada a defesa prévia à Delegatária, no respectivo processo, no prazo de cinco dias úteis. (Vide art. 4º da Lei nº 21.121, de 03/01/2014.)

\item As multas do Sistema Intermunicipal de Passageiros serão calculadas, desprezando-se os centavos, em função do coeficiente tarifário, do piso I para o serviço convencional e as multas do Sistema Metropolitano de Passageiros serão calculadas, desprezando-se os centavos, em função do coeficiente tarifário metropolitano e ambas terão gradação, valores e o seu recolhimento de acordo com este Regulamento.

\begin{enumerate}[label= \S \arabic*] %parágrafo

\item As multas aplicadas pelo DER-MG deverão ser recolhidas, através do Documento de Arrecadação Estadual – DAE, emitido pelo próprio DER-MG.

\item Sobre os valores das multas recolhidas em atraso, pela Delegatária, incidirá a aplicação da taxa SELIC, a partir do vencimento das mesmas.

\end{enumerate}

\item As multas terão a seguinte gradação:

\begin{enumerate}[label=\roman*.]

\item quinhentas vezes o coeficiente tarifário;

\item mil vezes o coeficiente tarifário;

\item duas mil vezes o coeficiente tarifário;

\item três mil vezes o coeficiente tarifário; e

\item cinco mil vezes o coeficiente tarifário.

\end{enumerate}

\item A multa de quinhentas vezes o coeficiente tarifário será aplicada quando ocorrer pelo menos uma das seguintes infrações:

\begin{enumerate}[label=\roman*.]

\item inexistência ou má condição de funcionamento e conservação do veículo, de equipamento obrigatório e do exigido para cada linha;

\item transporte de bagagem ou encomenda fora do lugar apropriado;

\item manutenção em serviço, para atendimento ao usuário, de pessoal não uniformizado ou sem identificação;

\item recusa de transporte de bagagem nos limites estabelecidos;

\item deixar de manter de forma visível no interior do veículo avisos determinados pela SETOP ou DER/MG;

\item conduzir o veículo sem os documentos de porte obrigatório definidos pela SETOP;

\item recusa à adoção de modelos de documentos padronizados pela SETOP e demais órgãos públicos.

\item atraso ou falta de encaminhamento à SETOP ou DER/MG de qualquer comunicação prevista neste Regulamento.

\item deixar de prestar ao usuário, quando solicitado, informações sobre o serviço;

\item não conter indicação dos pontos extremos da linha na parte dianteira externa do veículo;

\item não assegurar ao passageiro o seu lugar, conforme a especificação do serviço da linha;

\item não atender dispositivo legal sobre reserva de assentos;

\item não zelar pela boa ordem no interior do veículo;

\item não prestar os esclarecimentos solicitados pelos agentes fiscais ou não permitir seu acesso ao interior do veículo;

\item não estacionar o veículo no horário e pelo tempo determinado pela SETOP o início da viagem;

\item deixar de manter afixado no PC o quadro de horário atualizado;

\item conduzir o veículo com as luzes internas apagadas, fora do perímetro urbano no horário do pôr do sol até o nascer do sol;

\item conduzir, do pôr do sol até o nascer do sol, com letreiro apagado; e

\item deixar o veículo do sistema metropolitano, para o início da viagem, com a porta de embarque fechada e do pôr do sol até o nascer do sol com as luzes internas e letreiro apagadas.
\end{enumerate}

\begin{enumerate}[label= \S \arabic*] %parágrafo

\item Para o Sistema Intermunicipal de Passageiros aplicam-se as penalidades dos incisos I a XV.

\item Para o Sistema Metropolitano de Passageiros aplicam-se as penalidades dos incisos I a XIV e XVI a XIX.

\end{enumerate}

\item A multa de mil vezes o coeficiente tarifário será imposta quando ocorrer pelo menos uma das seguintes infrações:

\begin{enumerate}[label=\roman*.]

\item conduta inconveniente do pessoal em serviço;

\item desrespeito ou oposição à fiscalização;

\item faltar com respeito com o usuário;

\item apresentação do veículo para início de viagem em más condições de funcionamento, conservação ou higiene;

\item receber bagagem cujo transporte seja vedado neste Regulamento;

\item deixar de auxiliar ou de controlar o embarque ou desembarque dos passageiros e de suas bagagens;

\item utilizar na limpeza interna dos veículos, substâncias que prejudiquem o conforto e coloque em risco a segurança;

\item a tripulação fumar no interior do veículo;

\item deixar de solicitar auxilio da autoridade competente, quando necessário, no caso de ocorrência de anormalidade;

\item manter conversa estando o veículo em movimento, exceto para prestar informações;

\item permitir no interior do veículo, comércio ou mendicância;

\item interromper a viagem ou abandonar o veículo ou posto de trabalho sem causa justificada;

\item dificultar ou obstruir o acesso da fiscalização às instalações da empresa;

\item transportar o passageiro sem efetuar a cobrança da respectiva passagem;

\item embarcar o passageiro em local não autorizado ou permitido;

\item retardar o horário de partida, exceto se o atraso não tiver sido causado pela delegatária;

\item cobrar, a qualquer título, importância não prevista neste Regulamento ou em lei específica;

\item não manter atualizado os dados cadastrais da empresa e dos veículos;

\item não substituir o veículo retirado de circulação;

\item não afixar, no interior do veículo, em local visível, o número do telefone e o endereço eletrônico de atendimento ao usuário para observações sobre o serviço prestado;

\item não favorecer o embarque e desembarque do passageiro;

\item transportar passageiro sem o bilhete de passagem;

\item reter a via do bilhete da passagem destinada ao passageiro;

\item alterar a capacidade do veículo, em desacordo com a ficha de registro;

\item apresentar o veículo, para início de viagem ou após os pontos de parada, sem condições de utilização;

\item utilizar veículo em serviço com lay-out diferente do apresentado à SETOP;

\item deixar de etiquetar a bagagem a ser acondicionada no bagageiro ou não devolvê-la ao portador da etiqueta;

\item não manter reserva de lugar para fiscalização, na forma prevista neste regulamento.

\end{enumerate}

\begin{enumerate}[label= \S \arabic*] %parágrafo

\item Para o Sistema Metropolitano de Passageiros aplicam-se as penalidades dos incisos I a XXI.

\item Para o Sistema Intermunicipal de Passageiros aplicam-se as penalidades dos incisos I a XXVIII.

\end{enumerate}

\item Será aplicada a multa de duas mil vezes o coeficiente tarifário, se ocorrer uma das seguintes infrações:

\begin{enumerate}[label=\roman*.]

\item emissão ou preenchimento de bilhete de passagem em desacordo com os padrões e valores estabelecidos;

\item recusar devolução de valor da passagem, em caso de desistência ou da não prestação do serviço, como previsto neste Regulamento;

\item recusar venda de passagem sem motivo justo;

\item transportar o auxiliar de viagem, o cobrador ou qualquer outra pessoa na cabine do veículo ou na escada do veículo, quando não houver lugar a ele reservado neste espaço;

\item permanecer com o veículo em serviço não aprovado pela SETOP;

\item suspensão parcial ou total do serviço em desacordo com este Regulamento;

\item recusa ou atraso no fornecimento de qualquer informação solicitada pela SETOP;

\item atrasar o pagamento da indenização, por dano ou extravio da bagagem, por mês de atraso

\item utilizar veículo não registrado na SETOP.

\item recusar embarcar ou desembarcar passageiros nos pontos aprovados sem motivo justificado;

\item recusar a dar prioridade ao transporte de bagagem do passageiro;

\item falta de assistência ao passageiro e à tripulação, em caso de acidente, avaria mecânica ou interrupção de viagem;

\item vender mais de um bilhete de passagem em duplicidade.

\item descumprir o itinerário, horário de partida, secionamento, restrições de seção, pontos de parada, ponto de embarque ou desembarque e ponto de controle determinado para realização da viagem.

\item efetuar baldeação em desacordo com este Regulamento.

\end{enumerate}

\begin{enumerate}[label= \S \arabic*] %parágrafo

\item Para o Sistema Metropolitano de Passageiros aplicam-se as penalidades dos incisos I a XIV.

\item Para o Sistema Intermunicipal de Passageiros aplicam-se as penalidades dos incisos I a XV.

\end{enumerate}

\item Será aplicada multa de três mil vezes o coeficiente tarifário quando ocorrer uma das seguintes infrações:

\begin{enumerate}[label=\roman*.]

\item transporte de passageiro além do limite estabelecido;

\item manutenção de motorista em serviço além da jornada legalmente permitida;

\item transportar passageiros em pé, à exceção dos casos previstos na legislação específica;

\item não apresentar veículo para vistoria, em data, horário e local estabelecido pelo DER/MG;

\item conduzir o veículo em condições que comprometam a segurança dos usuários e demais condutores da via;

\item utilizar veículo com lacre ou instrumentos de controle de passageiros danificado, violado ou adulterado;

\item colocar ou manter em serviço veículo sem condições de segurança;

\item apresentar sintomas de embriaguez durante sua jornada de trabalho;

\item não iniciar os serviços no prazo fixado pela SETOP, ou suspender a prestação dos serviços em desacordo com este Regulamento;

\item venda de passagem para ponto de seção ou para local que não constar do Quadro de Regime de Funcionamento da linha;

\item cancelamento de viagem quando já houver sido efetuada venda de passagem.

\end{enumerate}

\begin{enumerate}[label= \S \arabic*] %parágrafo

\item Para o Sistema Metropolitano de Passageiros aplicam-se as penalidades dos incisos I a IX.

\item Para o Sistema Intermunicipal de Passageiros aplicam-se as penalidades dos incisos I a XI.

\end{enumerate}

\item Poderá ser aplicada, pelo Subsecretário de Estado de Transportes, advertência escrita à Delegatária que cometer falta grave, acompanhada de multa de cinco mil vezes o coeficiente tarifário, do Sistema Intermunicipal de Passageiros, da tabela referente ao piso tipo I para o serviço convencional.

Parágrafo único. As multas aplicadas pela SETOP deverão ser recolhidas, através do Documento de Arrecadação Estadual – DAE, emitido pela própria Secretaria.

\item São consideradas faltas graves:

\begin{enumerate}[label=\roman*.]

\item executar serviço regular não autorizado pela SETOP;

\item paralisar o serviço sem prévia autorização da SETOP;

\item perder as condições econômicas, técnicas ou operacionais para manter a adequada prestação do serviço delegado;

\item não atender intimação do DER-MG para regularizar a prestação do serviço;

\item não atender solicitação de atualização de dados cadastrais junto à SETOP, no prazo de dez dias, sem justificativa devida; e

\item não recolher ao DER-MG, por período superior a sessenta dias, os valores referentes ao Custo de Gerenciamento do Sistema de Transporte Coletivo Metropolitano – CGO, a Taxa de Gerenciamento do Sistema de Transporte Intermunicipal – TGO e multas.

\end{enumerate}

Parágrafo único. As faltas graves deverão ser apuradas em regular processo administrativo, por comissão designada pelo Subsecretário de Estado de Transportes, assegurado o direito de ampla defesa.

\item A Delegatária autuada recolherá ao DER-MG a quantia relativa ao valor da multa aplicada, no prazo de dez dias úteis, contados da publicação da decisão definitiva.

\item A penalidade de declaração de inidoneidade poderá ser aplicada à Delegatária, quando, em razão dos contratos firmados:

\begin{enumerate}[label=\roman*.]

\item apresentar falsa denúncia, dado falso ou documento adulterado, em proveito próprio ou prejuízo de outro; ou

\item tiver sofrido condenação definitiva por praticar, por meios dolosos, fraude fiscal no recolhimento de quaisquer tributos.

\end{enumerate}

Parágrafo único. A pena de declaração de inidoneidade pode ser aplicada pelo Secretário de Estado de Transportes e Obras Públicas, assegurada a defesa prévia à Delegatária, no respectivo processo, no prazo de dez dias corridos contados da abertura de vista, podendo a reabilitação ser requerida decorridos dois anos de sua aplicação.

\end{enumerate}

\section{Dos Procedimentos para Aplicação das Penalidades}

\begin{enumerate}[resume, label=Art. \arabic*]

\item O procedimento para aplicação da penalidade de multa, prevista no art. 94, I, deste Regulamento, terá início com o auto de infração, lavrado por agente fiscal, quando a infração for constatada e conterá, conforme o caso:

\begin{enumerate}[label=\roman*.]

\item identificação da Delegatária;

\item identificação da linha, número de ordem ou placa do veículo, quando for o caso;

\item local, data e hora da infração;

\item designação do infrator;

\item infração cometida e o dispositivo legal, regulamentar ou contratual violado;

\item assinatura do agente fiscal e respectivo número de identificação.

\end{enumerate}

\begin{enumerate}[label= \S \arabic*] %parágrafo

\item A lavratura do auto de infração pelo agente fiscal far-se-á em pelo menos duas vias de igual teor, devendo o infrator ou seu preposto, quando for o caso, apor o "ciente" no auto.

\item O ciente no auto de infração pelo infrator não significa o reconhecimento da falta, assim como a sua recusa em assiná-lo não o invalida.

\item O auto lavrado não poderá ser inutilizado nem sustada a sua tramitação, devendo o agente fiscal remetê-lo à autoridade competente, ainda que tenha cometido erro ou engano no preenchimento, com a devida justificativa para que seja cancelado.

\item O auto deverá ser registrado no sistema do DER-MG.

\end{enumerate}

\item A instrução do processo do auto de infração será realizada por comissão constituída de pelo menos três servidores, designados em ato do Diretor de Fiscalização do DER-MG.

\end{enumerate}