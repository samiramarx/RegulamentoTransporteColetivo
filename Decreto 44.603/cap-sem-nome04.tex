\chapter{.}

\section{Da Criação de Linha}

\begin{enumerate}[resume, label=Art. \arabic*]

\item As premissas para a criação de uma linha, por iniciativa da SETOP ou por solicitação do interessado, deverão atender aos seguintes requisitos:

\begin{enumerate}[label=\roman*.]

\item intercâmbio entre os pontos extremos no contexto econômico e social da região;

\item capacidade de geração de transporte nas localidades a serem atendidas;

\item caráter de permanência da ligação, em função do interesse público e de sua viabilidade econômica; e

\item inexistência de possibilidade de prejuízo ou desequilíbrio econômico-financeiro de outros serviços já existentes.

\end{enumerate}

\end{enumerate}

\section{Da Delegação}

\begin{enumerate}[resume, label=Art. \arabic*]

\item  O serviço de transporte coletivo de passageiros será executado diretamente pela SETOP ou delegado a terceiros, observada a legislação específica.

\item O serviço de transporte coletivo de passageiros compreende todos os veículos, equipamentos, instalações e atividades inerentes a sua execução.

\item Incumbe à Delegatária a execução do serviço, por sua conta e risco, respondendo por todos os prejuízos causados ao usuário ou a terceiros, não sendo imputável à SETOP qualquer responsabilidade, direta ou indireta.

\begin{enumerate}[label= \S \arabic*] %parágrafo

\item A fiscalização exercida pelo DER-MG não exclui ou atenua a responsabilidade da Delegatária.

\item A Delegatária se obriga a prestar os serviços, de acordo com o presente Regulamento, nos termos da legislação e normas pertinentes.

\item É de exclusiva responsabilidade da Delegatária, o recrutamento, a seleção, a admissão e demais providências administrativas referentes ao pessoal que contratar, respondendo pelos encargos trabalhistas e previdenciários.

\item As contratações feitas pela Delegatária, inclusive de mão-de-obra, são de sua exclusiva responsabilidade e regidas pelas disposições de direito privado e pela legislação trabalhista, não se estabelecendo quaisquer relações ou vínculos entre os terceiros contratados, o DER-MG e a SETOP.

\item A Delegatária deverá comunicar à SETOP, por escrito, no prazo de dez dias, a contar de qualquer incidente que interfira ou impeça a boa execução dos serviços delegados, ou que contrarie as normas regulamentares vigentes, por motivo de força maior.

\item A Delegatária será direta e solidariamente responsável pelo comportamento e eficiência do pessoal sob sua direção.

\item A Delegatária se obriga a facilitar ao DER-MG todos os meios necessários à fiscalização dos serviços contratados, bem como a sua ação específica, relativa à operação dos serviços.

\item A Delegatária deverá cumprir os procedimentos de proteção ambiental, responsabilizando-se pelos danos causados ao meio ambiente, por ação ou omissão, decorrentes de sua culpa ou dolo, durante a execução da Delegação, nos termos da legislação pertinente.

\end{enumerate}

\end{enumerate}

\section{Da Cessão da Delegação}

\begin{enumerate}[resume, label=Art. \arabic*]

\item A transferência da delegação ou do controle societário da Delegatária dependerá da prévia anuência da SETOP, sob pena de caducidade da delegação, observado o art. 27, da Lei Federal nº 8.987, de 1995.

\begin{enumerate}[label= \S \arabic*] %parágrafo

\item Para fins da obtenção da anuência de que trata o caput, o pretendente deverá:

\begin{enumerate}[label=\roman*.]

\item atender às exigências de capacidade técnica, idoneidade financeira e regularidade jurídica e fiscal exigidas em legislação específica; e

\item comprometer-se a cumprir integralmente as obrigações da delegação com a SETOP, bem como as disposições deste Regulamento e demais legislação aplicada.

\end{enumerate}

\item A empresa Delegatária deverá comunicar à SETOP qualquer alteração em seu contrato social ou em seus estatutos, no prazo de dez dias a contar do registro na Junta Comercial ou em repartição competente.

\end{enumerate}

\end{enumerate}

\section{Da Extinção da Delegação}

\begin{enumerate}[resume, label=Art. \arabic*]

\item Extingue-se a delegação por:

\begin{enumerate}[label=\roman*.]

\item advento do termo da delegação;

\item encampação;

\item caducidade;

\item rescisão;

\item anulação; e

\item falência ou extinção da Delegatária e falecimento ou incapacidade do titular, no caso de empresa individual.

\end{enumerate}

\begin{enumerate}[label= \S \arabic*] %parágrafo

\item Extinta a delegação, retornarão à SETOP todos os direitos e privilégios transferidos à Delegatária.

\item O único bem reversível é o direito de exploração comercial das linhas de transporte coletivo de passageiros.

\item Extinta a delegação, haverá a imediata assunção do serviço pela SETOP, procedendo-se aos levantamentos, avaliações e liquidações necessárias.

\item Nos casos previstos nos incisos I e II do art. 75, a SETOP, antecipando-se à extinção da delegação, procederá aos levantamentos e avaliações necessários à determinação dos montantes da indenização que será devida à Delegatária, na forma dos arts 36 e 37 da Lei Federal nº 8.987, de 1995, descontados os valores devidos e os danos causados pela Delegatária.

\end{enumerate}

\item A reversão no advento do termo da delegação far-se-á mediante a indenização das parcelas de investimentos vinculados ao bens reversíveis ainda não amortizados ou depreciados, que tenham sido realizados com o objetivo de cumprir os compromissos da delegação.

\item Considera-se encampação a retomada do serviço pela SETOP durante o prazo da delegação, por motivo de interesse público, mediante lei autorizativa específica e após prévio pagamento da indenização, na forma do art.76.

\item A inexecução total ou parcial da delegação acarretará, a critério da SETOP, a declaração de sua caducidade ou a aplicação das sanções regulamentares estabelecidas neste regulamento, bem como previsto no art. 38 da Lei Federal nº 8.987, de 1995.

\begin{enumerate}[label= \S \arabic*] %parágrafo

\item A declaração de caducidade da delegação deverá ser precedida da apuração da inadimplência da Delegatária em processo administrativo, assegurado o direito de ampla defesa.

\item Não será instaurado processo administrativo de inadimplência contra a Delegatária, sem que a mesma seja devidamente instada pela SETOP a sanar as falhas apontadas, no prazo de 10 (dez) dias, contados a partir do recebimento da referida comunicação pela Delegatária.

\item Instaurado processo administrativo e comprovada a inadimplência, a caducidade será declarada pelo Secretário de Estado de Transportes e Obras Públicas, por meio de despacho fundamentado que será publicado no Órgão Oficial dos Poderes do Estado, após conclusão do referido processo independente de indenização prévia, calculada no decurso do processo.

\end{enumerate}

\item Declarada a caducidade, não advirá para a SETOP qualquer espécie de responsabilidade em relação aos encargos, ônus, obrigações ou compromissos com terceiros ou com empregados da Delegatária.

\item A Delegação poderá ser rescindida por iniciativa da Delegatária, no caso de descumprimento das normas contratuais pela SETOP, mediante ação judicial especialmente intentada para este fim.

Parágrafo único. Na hipótese prevista no caput, os serviços prestados pela Delegatária não poderão ser interrompidos ou paralisados, até a decisão judicial transitada em julgado.

\item A Delegação poderá ser rescindida pela SETOP nos casos previstos na Lei Federal nº 8.666, de 1993.




\end{enumerate}






