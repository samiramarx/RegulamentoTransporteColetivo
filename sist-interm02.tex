\chapter{DO REGIME DE FUNCIONAMENTO DE LINHA DO SISTEMA INTERMUNICIPAL DE PASSAGEIROS}

\begin{enumerate}[resume, label=Art. \arabic*]

\item A fixação e a alteração do regime de funcionamento de linhas ou das especificações de serviços serão estabelecidas pela SETOP, por sua iniciativa ou mediante solicitação da Delegatária ou de terceiros preservando as linhas próprias do trecho, e constarão do novo Quadro de Regime de Funcionamento – QRF da linha.

\item A SETOP poderá, por um período de noventa dias, visando sanar irregularidades de operação e atender aos interesses dos passageiros, operar diretamente ou convocar outras Delegatárias que apresentem frota e pessoal disponíveis para operar a linha.

\begin{enumerate}[label= \S \arabic*] %parágrafo

\item As Delegatárias convocadas serão remuneradas pela receita auferida pela operação das linhas, no período de convocação.

\item Não sendo sanadas as irregularidades a que se refere o caput, durante o período da intervenção será aberto processo administrativo, visando à caducidade da concessão.

\end{enumerate}

\item Ficam estabelecidos os seguintes padrões de serviços para as linhas de ônibus:

\begin{enumerate}[label=\roman*.]

\item convencional;

\item convencional executivo;

\item comercial;

\item comercial executivo;

\item leito; e

\item semi-leito.

\end{enumerate}

Parágrafo único. A SETOP disciplinará a metodologia de remuneração de cada um dos padrões de serviços considerando a classificação das rodovias quanto à superfície de rolamento e o tipo de veículo utilizado.

\item A Delegatária do serviço será obrigada a comunicar à SETOP, no prazo máximo de dez dias a contar da ocorrência registrada durante a viagem, fato que implicar alteração do regime de funcionamento da linha ou especificação do respectivo serviço.

\begin{enumerate}[label= \S \arabic*] %parágrafo

\item Em caso de ocorrência de acidentes com vítimas, a Delegatária será obrigada a prestar imediata e adequada assistência ao usuário, devendo a SETOP e o DER-MG serem comunicados a respeito, até o primeiro dia útil após o acidente.

\item O Boletim de Ocorrência ou Laudo Técnico Pericial deverá ser encaminhado ao DER-MG no prazo máximo de dez dias a contar da data da sua disponibilidade pelo órgão emitente.

\end{enumerate}

\item A indicação dos pontos de parada é de responsabilidade da Delegatária e deverá obedecer aos seguintes critérios:

\begin{enumerate}[label=\roman*.]

\item os pontos de parada deverão estar, obrigatoriamente, situados no itinerário da linha;

\item o tempo de viagem entre dois pontos consecutivos deverá ser de três horas e trinta minutos, no máximo; e

\item os pontos de parada deverão atender às condições técnicas, operacionais e de conforto, higiene e segurança do usuário.

Parágrafo único. Os pontos de paradas indicados pela Delegatária serão previamente vistoriados pelo DER-MG e, após liberação pela SETOP, deverão constar no QRF da linha.

\end{enumerate}

\item A frota e a freqüência da linha serão estabelecidas pela SETOP em função da demanda, do nível de conforto dos passageiros, da segurança do tráfego, da velocidade operacional e da extensão do itinerário.


\item É vedada a imposição de restrição de Seção de linha ou de serviços já existentes.

Parágrafo único. A SETOP poderá estabelecer a restrição de Seção, objetivando a manutenção do equilíbrio econômico-financeiro do sistema, nos casos abaixo:

\begin{enumerate}[label=\roman*.]

\item criação de novo serviço; e

\item alteração de regime de funcionamento de linha ou de serviços já existentes, inclusive no remanejamento de horários.

\end{enumerate}

\item É vedado à Delegatária vender passagens para localidade que não conste como Seção dos Quadros de Regime de Funcionamento da linha.

\item É vedada a implantação de ponto de Seção situado a menos de dez quilômetros de outro já existente na linha.

\item O transbordo ou baldeação de passageiros poderá ser realizado, e deverá constar do Quadro de Regime de Funcionamento da linha, excetuando-se os casos de emergência.

\item A Delegatária poderá realizar viagens de reforço nos serviços existentes da linha, por necessidade do serviço e sem caráter habitual, devendo estas ser declaradas no Quadro Demonstrativo do Movimento de Passageiros.

Parágrafo único. A viagem de reforço deverá cumprir as mesmas especificações previstas nos QRF para o horário a ser reforçado.

\item O atendimento parcial deverá ser realizado estritamente no itinerário da linha, não podendo ser objeto de fusão, prolongamento ou alteração de itinerário.

\begin{enumerate}[label= \S \arabic*] %parágrafo

\item O atendimento parcial só poderá ser realizado por Delegatária de linha que tenha ponto de Seção na localidade a ser atendida.

\item A linha de menor percurso deverá ser preservada sempre que for implantado um atendimento parcial.

\end{enumerate}

\item A Delegatária poderá cancelar horários regulares da linha em época de baixa demanda, declarando-os expressamente no QDMP.

\begin{enumerate}[label= \S \arabic*] %parágrafo

\item A viagem deverá ser obrigatoriamente realizada com qualquer número de passageiros, caso a venda de passagem já tenha sido efetuada.

\item A suspensão provisória de um mesmo horário por mais de vinte vezes consecutivas deverá ser comunicada à SETOP com antecedência de cinco dias da viagem programada.

\item A qualquer tempo, o horário suspenso, conforme mencionado no § 2º, poderá retornar mediante nova comunicação à SETOP com antecedência de cinco dias da viagem programada.

\end{enumerate}

\item Os pontos extremos, pontos de Seção e os pontos de parada deverão estar, sempre que possível, localizados nos terminais rodoviários.


\item A SETOP poderá autorizar conexão de linhas, a pedido da Delegatária ou por sua própria iniciativa, no interesse do serviço.


\item A fusão de linhas será admitida quando for assegurado o atendimento às localidades dos itinerários das linhas envolvidas.

\begin{enumerate}[label= \S \arabic*] %parágrafo

\item A fusão será solicitada à SETOP pela Delegatária.

\item O serviço resultante de fusão de linhas intermunicipais, não poderá ser objeto de prolongamento, encurtamento ou alteração de itinerário, podendo ser cancelado a qualquer tempo, por solicitação da Delegatária ou por iniciativa da SETOP, retornando à sua condição original.

\item Os novos pontos extremos não poderão ser coincidentes com serviço existente.

\end{enumerate}

\item Poderá haver o prolongamento de linha quando:

\begin{enumerate}[label=\roman*.]

\item a distância entre o ponto extremo original e o pretendido não for superior, em nenhuma hipótese, a vinte por cento da quilometragem entre os pontos extremos primitivos estabelecidos no contrato de delegação, não computadas as distâncias dos pontos de Seção fora do eixo do itinerário;

\item os novos pontos extremos não forem coincidentes com serviço existente; e

\item não causar concorrência ruinosa a serviço existente.

\end{enumerate}

\begin{enumerate}[label= \S \arabic*] %parágrafo

\item A alteração de pontos extremos dentro do mesmo município não caracteriza prolongamento, devendo obedecer aos seguintes parâmetros:

\begin{enumerate}[label=\roman*.]

\item carência de transporte intermunicipal em face da efetiva demanda; e

\item impossibilidade de o município implantar atendimento próprio.

\end{enumerate}

\item O serviço resultante de alteração de pontos extremos dentro do mesmo município, por ser precário, não poderá ser objeto de prolongamento ou alteração de itinerário, podendo ser cancelado a qualquer tempo, por solicitação da Delegatária, por iniciativa da SETOP ou quando o município implantar serviço próprio, retornando à sua condição original.

\end{enumerate}

\item Será permitido o encurtamento de linha, nos seguintes casos:

\begin{enumerate}[label=\roman*.]

\item os novos pontos extremos não forem coincidentes com os de outra linha;

\item não prejudicar os serviços existentes; e

\item a localidade indicada como novo ponto extremo ou ponto de Seção for ponto de Seção da linha encurtada.

\end{enumerate}

\item A SETOP poderá alterar o padrão de serviço das linhas em alguns horários ou em sua totalidade, estabelecendo, para cada caso, regime de funcionamento e tarifa próprios.

\item A alteração de itinerário de uma linha será admitida para proporcionar maior economia, conforto ou segurança ao usuário, nas seguintes condições:

\begin{enumerate}[label=\roman*.]

\item nova rodovia ou trecho com melhores condições de tráfego for implantada ou pavimentada;

\item não houver prejuízo ou desequilíbrio econômico-financeiro de outros serviços; ou

\item o objetivo principal não for o atendimento do mercado intermediário.

\end{enumerate}

\begin{enumerate}[label= \S \arabic*] %parágrafo

\item A alteração de itinerário obriga a Delegatária a atender, também, o acréscimo de serviço e, sendo o caso, a manter o serviço que vinha prestando no antigo itinerário.

\item Não será admitido um terceiro itinerário para cada linha, respeitadas as situações existentes na data de publicação deste Regulamento.

\item Poderá ser implantado itinerário alternativo e provisório quando necessário, por interrupção do itinerário existente, mantido o atendimento existente no quadro de regime de funcionamento do itinerário oficial.

\end{enumerate}

\item O aviso contendo a proposta e a decisão posterior relativas às alterações no regime de funcionamento das linhas, inclusive de atendimento parcial, conexão, fusão, prolongamento, encurtamento de linhas, alteração de padrão de serviço e alteração de itinerário serão publicados no Órgão Oficial dos Poderes do Estado, podendo ser cancelados, cessados os motivos que os determinaram.

Parágrafo único. Contra as alterações propostas no aviso, previstas no caput, caberão impugnações ou manifestações, no prazo de dez dias corridos, a contar do primeiro dia útil após a data de publicação no Órgão Oficial dos Poderes do Estado.

\end{enumerate}
















