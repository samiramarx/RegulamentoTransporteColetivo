\chapter{DOS ENCARGOS DA DELEGATÁRIA}

\begin{enumerate}[resume, label=Art. \arabic*]

\item São obrigações da Delegatária:

\begin{enumerate}[label=\roman*.]

\item executar os serviços da linha de ônibus na forma deste Regulamento e legislação pertinente;

\item transportar com segurança os passageiros, suas bagagens e encomendas;

\item responder por todos os prejuízos, que no exercício da delegação, cause aos passageiros e a terceiros;

\item responsabilizar-se pelo pagamento de encargos fiscais, tributários, previdenciários, trabalhistas e sociais resultantes da delegação;

\item iniciar os serviços no prazo fixado pela SETOP em exato cumprimento às especificações do serviço delegado;

\item cumprir o itinerário, horário de partida, secionamento, restrições de Seção, pontos de parada, pontos de embarque e desembarque e pontos de controle;

\item adotar as tarifas fixadas para o serviço estabelecidas pela SETOP;

\item indenizar ao passageiro a bagagem etiquetada extraviada ou danificada, na forma deste Regulamento;

\item preencher corretamente o documento exigido pela SETOP para a operação da linha ou serviço;

\item estacionar o veículo para o início da viagem, no horário determinado pela SETOP;

\item respeitar o tempo previsto nos pontos de parada;

\item apresentar o veículo limpo, interna e externamente, para o início da viagem;

\item adotar modelo de impresso determinado pela SETOP e demais órgãos públicos do Estado;

\item reservar nas viagens um lugar para a fiscalização do DER-MG, até seis horas antes do início de cada viagem;

\item fornecer todas as informações solicitadas pela SETOP no prazo determinado;

\item comunicar ao DER-MG, no prazo de dez dias, a contar da ocorrência de qualquer incidente no serviço, para o sistema intermunicipal, devidamente instruído;

\item reembolsar o passageiro o valor da passagem não utilizada ou revalidá-la, se apresentada até doze horas antes do início da viagem;

\item – manter os dados cadastrais atualizados na SETOP;

\item recolher, no prazo determinado, quantia devida à SETOP e ao DER-MG a qualquer título;

\item prestar serviço até sessenta dias após a decisão definitiva de paralisação ou cancelamento do objeto da delegação;

\item providenciar o desembarque dos passageiros, caso o veículo tenha que estacionar em local que não ofereça condições de segurança;

\item apresentar o veículo para vistoria, quando solicitado pelo DER-MG, em data, horário e local estabelecidos;

\item afixar os quadros de horários atualizados das linhas metropolitanas da RMBH em local visível, nos pontos de controle;

\item manter no interior do veículo, de forma visível, as informações e avisos determinados pela SETOP;

\item portar no veículo em operação os documentos de porte obrigatório conforme a legislação vigente;

\item fornecer as informações previstas no QRF e no MCO;

\item permitir o acesso dos agentes fiscais aos veículos e às instalações da empresa;

\item substituir imediatamente o veículo retirado de circulação;

\item comunicar à SETOP, toda e qualquer alteração do contrato social, no prazo estabelecido neste Regulamento;

\item preservar a inviolabilidade do instrumento de controle de passageiros no veículo e outros dispositivos estabelecidos pela SETOP e mantê-los em perfeitas condições de uso;

\item utilizar o veículo em serviço na linha devidamente identificado e na padronização apresentada à SETOP;

\item realizar o transbordo de passageiros nos casos emergenciais ou previstos no QRF ou no QCO da linha;

\item manter em operação somente veículo devidamente cadastrado na SETOP;

\item XXXIV – manter a tripulação devidamente uniformizada;

\item afixar em local visível no interior do veículo o número do telefone ou endereço eletrônico para atendimento ao usuário;

\item respeitar e fazer cumprir todos os direitos dos usuários;

\item permitir e facilitar o levantamento de informações e a realização de estudos por pessoal credenciado pela SETOP e DER-MG;

\item manter em funcionamento locais de venda de passagens em horários compatíveis com os horários das linhas; e

\item não veicular publicidade ou prestar informações duvidosas que possam induzir o usuário a erro.

\end{enumerate}

\end{enumerate}

\section{Dos Encargos dos Prepostos da Delegatária}

\begin{enumerate}[resume, label=Art. \arabic*]

\item O preposto deverá:

\begin{enumerate}[label=\roman*.]

\item manter-se em adequado estado de asseio, limpeza e higiene;

\item prestar informação ao passageiro relativa à operação dos serviços;

\item zelar pela boa ordem no interior do veículo;

\item entregar à administração da Delegatária objeto encontrado no veículo após a realização da viagem;

\item impedir o acesso ao veículo e recusar transporte ao passageiro que estiver em visível estado de embriaguez ou sob efeito de substância tóxica de qualquer natureza que possa comprometer a segurança, higiene, saúde pública, conforto ou a tranqüilidade dos demais passageiros;

\item não reter a via do bilhete de passagem destinada ao passageiro;

\item impedir a prática de comércio ambulante e de mendicância dentro do veículo;

\item solicitar auxílio e colaborar com a autoridade competente no caso de anormalidade;

\item permitir, facilitar e auxiliar o pessoal da SETOP e do DER-MG na realização de estudo ou fiscalização;

\item conduzir-se com decoro, urbanidade e respeito ao público;

\item manter em bom estado de conservação e à disposição dos agentes fiscais, todos os documentos de porte obrigatório nos veículos;

\item providenciar o desembarque dos passageiros, com segurança, caso o veículo necessite ser imobilizado;

\item acatar as determinações da SETOP e do DER-MG; e

\item advertir o passageiro quanto à proibição de fumar no interior do veículo.

\end{enumerate}

\item Ao preposto é vedado:

\begin{enumerate}[label=\roman*.]

\item recusar a venda de passagem sem motivo justo;

\item efetuar qualquer modalidade de comércio não-autorizado de bilhete de passagem;

\item desacatar ou desrespeitar a fiscalização;

\item trabalhar em estado de embriaguez ou sob efeito de substância tóxica de qualquer natureza;

\item transportar passageiro além da capacidade do veículo;

\item permitir o transporte de passageiros ou prepostos na cabine, nas escadas de acesso ao interior dos veículos, desde o início até o fim das viagens, salvo quando o veículo possuir assento destinado ao auxiliar de viagem, com utilização do cinto de segurança;

\item fazer uso de aparelhos sonoros durante a operação do serviço e no interior de veículo, à exceção de aparelho de intercomunicação e música ambiente autorizados;

\item fumar no interior do veículo;

\item abandonar o veículo ou posto de trabalho, sem causa justificada; e

\item omitir informação sobre irregularidade de que tenha conhecimento, no exercício de suas funções.

\end{enumerate}

\item O motorista deverá:

\begin{enumerate}[label=\roman*.]

\item conduzir o veículo de acordo com as normas de trânsito;

\item auxiliar, em caso de interrupção de viagem, a condução do passageiro a outro veículo;

\item conduzir o veículo do Sistema Metropolitano, do pôr do sol até o nascer do sol, com as luzes internas acesas no perímetro urbano;

\item IV – conduzir o veículo, do pôr do sol até o nascer do sol, com letreiro aceso;

\item atender à solicitação de parada pelo agente fiscal, quando devidamente identificado;

\item aproximar o veículo da guia da calçada ou baia nos ponto de embarque e desembarque de passageiros, facilitando o acesso dos passageiros;

\item atender sinal de parada e não recusar passageiro no ponto demarcado, estando o veículo com sua lotação incompleta;

\item conduzir o veículo de forma a não comprometer a segurança do passageiro ou dos demais usuários da via;

\item conduzir o veículo em velocidade compatível com a via, sem provocar partidas, freadas ou conversões bruscas, prejudicando a condição de conforto e segurança dos passageiros;

\item prestar assistência imediata e adequada ao passageiro em caso de acidente;

\item providenciar transporte, refeição e hospedagem para o passageiro, nos casos previstos neste Regulamento; e

\item acatar as determinações do agente fiscal.

\end{enumerate}

\item Ao motorista é vedado:

\begin{enumerate}[label=\roman*.]

\item efetuar a partida do veículo sem que termine o embarque ou desembarque de passageiros;

\item interromper a viagem sem motivo justo;

\item conversar, com o veículo em movimento, exceto para prestar informações;

\item permitir o embarque ou desembarque de usuário pela porta indevida; e

\item movimentar o veículo sem que as portas de embarque e desembarque estejam fechadas.

\end{enumerate}

\item O auxiliar de viagem ou cobrador deverá:

\begin{enumerate}[label=\roman*.]

\item receber e etiquetar a bagagem que lhe for confiada pelo passageiro, zelando pela sua conservação até ser devolvida, no serviço convencional;

\item zelar para que a bagagem ou encomenda sejam transportadas no lugar apropriado, no serviço convencional;

\item impedir o uso, por parte do passageiro, de aparelho sonoro, salvo com utilização de fones de ouvidos;

\item auxiliar na operação de embarque e desembarque de passageiros;

\item efetuar a cobrança de passagem no local próprio junto ao instrumento de controle de passageiros, exceto nos veículos de característica rodoviária;

\item impedir que o passageiro viaje sem o respectivo bilhete de passagem;

\item auxiliar o motorista, em caso de acidente de trânsito envolvendo o veículo, providenciando atendimento e remoção da vítima, quando for o caso;

\item fornecer ao passageiro comprovante do pagamento da bagagem individual excedente;

\item efetuar a cobrança do preço de passagem na forma e no valor estabelecidos pela SETOP;

\item assegurar ao passageiro seu lugar no veículo;

\item acatar às determinações do agente fiscal;

\end{enumerate}

\item Ao auxiliar de viagem ou trocador é vedado:

\begin{enumerate}[label=\roman*.]

\item restringir o transporte da bagagem do passageiro a favor de encomenda, no serviço de característica rodoviária;

\item conversar com o motorista, quando em viagem, exceto para prestar informações relativas ao serviço;

\item emitir o bilhete de passagem em duplicidade ou em desacordo com as normas vigentes;

\item ocupar poltrona destinada aos passageiros, quando o veículo possuir assento junto à cabine do motorista; e

\item sonegar troco ao passageiro ou obter ganho indevido na cobrança do preço de passagem.

\end{enumerate}

\end{enumerate}