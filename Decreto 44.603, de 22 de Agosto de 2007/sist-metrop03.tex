\chapter{DAS CARACTERÍSTICAS OPERACIONAIS DA LINHA OU DO ATENDIMENTO COMPLEMENTAR DO SISTEMA METROPOLITANO DE PASSAGEIROS}

\begin{enumerate}[resume, label=Art. \arabic*]

\item A fixação e a alteração das especificações de serviços a serem executadas pelas linhas ou atendimentos complementares, serão estabelecidas pela SETOP, por sua iniciativa ou mediante solicitações de interessados, preservando as linhas ou atendimentos complementares próprias do trecho e suas áreas de atendimento.

Parágrafo único. As alterações de que trata o caput deverão constar de um novo QCO para a linha ou para o atendimento complementar.

\item A SETOP, visando sanar irregularidades de operação e atender aos interesses dos passageiros, poderá operar diretamente ou convocar outras Delegatárias que apresentem frota e pessoal disponíveis para operar a linha que apresentar irregularidade, por um período de até noventa dias.

\begin{enumerate}[label= \S \arabic*] %parágrafo

\item As Delegatárias convocadas serão remuneradas pela receita auferida pela operação das linhas, no período de convocação.

\item Não sendo sanadas as irregularidades citadas no art. 59, durante o período da intervenção será aberto processo administrativo, visando a caducidade da concessão.

\end{enumerate}

\item Para a fixação da ocupação máxima do veículo e do nível de serviço, a frota e os espaçamentos entre os horários das linhas ou atendimentos complementares serão calculados em função da demanda existente, do nível de conforto do passageiro, da segurança do tráfego, da velocidade operacional e da extensão do itinerário, conforme metodologia de programação operacional própria.


\item A frota reserva deverá estar disponível para substituir os veículos em operação, quando necessário.

\begin{enumerate}[label= \S \arabic*] %parágrafo

\item A quantidade de veículos que compõem a frota reserva deverá ser de no mínimo um veículo, e máximo dez por cento da frota especificada, por empresa.

\item O percentual máximo poderá ser aumentado pela empresa, não sendo computado na apuração da metodologia tarifária.

\item Os veículos que excederem o percentual máximo de dez por cento a que se refere o § 1º serão aqueles de idade cronológica mais avançada em relação à frota total da empresa cadastrada na SETOP.

\end{enumerate}

\item A frota empenhada para cada linha ou atendimento complementar será dimensionada para cada período distinto do dia.

Parágrafo único. Para o dimensionamento da frota empenhada serão levados em consideração o tempo de viagem -"ciclo"- e a quantidade de horários ou de partidas existentes no período em questão, constantes do QCO e das ordens de serviços.

\item A SETOP poderá requisitar veículo e pessoal de operação para atendimento a serviço de emergência ou de interesse público.


\item Para fins de fiscalização do cumprimento do quadro de horário especificado, as viagens deverão ser realizadas nos horários estabelecidos no QCO ou nas ordens de serviços, admitindo-se as seguintes tolerâncias:

\begin{enumerate}[label=\roman*.]

\item a antecipação ou atraso máximo igual ao intervalo especificado no QCO, quando o intervalo de tempo especificado no QCO ou ordens de serviços para a viagem for inferior a dez minutos; e

\item a antecipação ou atraso máximo de dez minutos, quando o intervalo de tempo especificado no QCO ou ordens de serviços para a viagem for superior a dez minutos.

\end{enumerate}

\begin{enumerate}[label= \S \arabic*] %parágrafo

\item A Delegatária poderá por necessidade de serviço e sem caráter habitual, realizar viagens suplementares, cumprindo as mesmas especificações dos serviços existentes da linha, devendo a mesma ser declarada no MCO e ou no QDMP.

\item A Delegatária poderá, em época de baixa demanda, cancelar horários regulares da linha, declarando-os expressamente no MCO ou no QDMP.

\end{enumerate}

\item A Delegatária ficará obrigada a comunicar à SETOP, no prazo máximo de dez dias, qualquer fato ocorrido durante a viagem que implicar em alteração do regime de funcionamento da linha ou especificação do respectivo serviço.

\begin{enumerate}[label= \S \arabic*] %parágrafo

\item Em caso de ocorrência de acidentes com vítimas, a Delegatária ficará obrigada a prestar imediata e adequada assistência ao usuário, devendo a SETOP e o DER-MG serem comunicados a respeito até o primeiro dia útil após o acidente.

\item O Boletim de Ocorrência ou Laudo Técnico Pericial deverá ser encaminhado ao DER-MG no prazo máximo de dez dias a contar da data da sua disponibilidade pelo órgão emitente.

\end{enumerate}

\item O atendimento complementar poderá ser implantado, desde que:

\begin{enumerate}[label=\roman*.]

\item complemente o atendimento na mesma área de captação e distribuição da linha existente que tenha necessidades diferenciadas;

\item interligue área de captação e distribuição da linha a pólos geradores de demanda metropolitana; e

\item promova o atendimento à área de captação e distribuição por diferentes corredores de transporte metropolitano.

\end{enumerate}

\item A implantação de Seção poderá ocorrer desde que não provoque desequilíbrio econômico-financeiro nos serviços existentes.

\item O itinerário e o ponto de controle das linhas ou atendimentos complementares poderão ser alterados, devendo-se respeitar suas funcionalidades, não sendo admitida alteração que venha a criar uma nova funcionalidade, excetuando-se o caso para atendimento a Pólos Geradores de Demandas Metropolitanas e Corredores de Transportes Metropolitanos.

\begin{enumerate}[label= \S \arabic*] %parágrafo

\item O Corredor de Transporte Regional poderá ser utilizado por linhas ou atendimentos complementares pertencentes às Áreas de Captação e Distribuição diferentes, desde que pertençam à mesma Bacia de Captação e Distribuição, e não tenham a mesma funcionalidade de outras linhas ou atendimentos complementares que já trafegam por este corredor.

\item O Corredor de Transporte Regional poderá ser utilizado, em caráter excepcional, por linhas ou atendimentos complementares pertencentes à outras Bacias de Captação e Distribuição, somente quando não houver disponibilidade de corredor metropolitano.

\item O Corredor de Transporte Metropolitano não poderá ser considerado como Área de Captação e Distribuição de uma Linha ou Atendimento Complementar, podendo ser utilizado por Linhas ou Atendimentos Complementares pertencentes a diferentes Bacias de Captação e Distribuição.

\item Um pólo gerador de demanda regional somente deverá ser atendido por linhas ou atendimentos complementares cujas Áreas de Captação e Distribuição pertençam à mesma Bacia onde esse se encontre situado, devendo essas linhas ou atendimentos complementares, após a saída de suas respectivas Áreas de Captação e Distribuição, somente se utilizarem dos Corredores de Transporte Regionais ou Metropolitanos para acessá-lo, podendo ser usadas Vias Locais caso sejam a única forma de acesso ao respectivo Pólo.

\item Um pólo gerador de demanda metropolitana poderá ser atendido por linhas ou atendimentos complementares pertencentes a Bacias diferentes, devendo, para tanto, serem utilizados os corredores de transporte metropolitanos para acessá-lo; poderão ser utilizados corredores de transporte regionais ou vias locais, se forem a única forma de acesso ao respectivo Pólo.

\item A Área de Captação e Distribuição de uma linha ou atendimento complementar que for cortada por uma outra linha ou atendimento complementar, cujo ponto final e respectiva Área de Captação e Distribuição sejam distintos, porém dependentes do sistema viário local para atingir o sistema coletor ou arterial, poderá ser atendida pela linha ou atendimento complementar cujo ponto de controle esteja situado em seu interior.

\end{enumerate}

\item O aviso contendo a proposta e o ato da decisão relativos à implantação de atendimento complementar, implantação de Seção, alteração de itinerário e de ponto de controle, serão publicadas no Órgão Oficial dos Poderes do Estado.

Parágrafo único. Contra as alterações propostas no aviso, previstas no caput, caberão impugnações ou manifestações, no prazo de dez dias corridos, a contar do primeiro dia útil após a data de publicação no Órgão Oficial dos Poderes do Estado.


\end{enumerate}








