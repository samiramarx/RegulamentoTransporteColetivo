\chapter{DISPOSIÇÕES PRELIMINARES}

\section{Do Serviço Adequado}

\begin{enumerate}[label=Art. \arabic*]
\item O transporte coletivo rodoviário intermunicipal e metropolitano realizado no território do Estado de Minas Gerais, é serviço público de competência da Secretaria de Estado de Transportes e Obras Públicas – SETOP, podendo ser prestado diretamente ou por delegação, nos termos da Lei Delegada nº 128, de 25 de janeiro de 2007, da Lei Delegada nº 164, de 25 de janeiro de 2007, da Lei Federal nº 8.666 de 21de junho de 1993, da Lei Federal nº 8.987, de 13de fevereiro de 1995, da Lei Federal nº 9.074, de 7 de julho de 1995, e da Lei Estadual nº 13.655, 14 de julho de 2000, e reger-se-á pelas normas deste Regulamento e legislação aplicável.

\item Considerar-se-á como serviço adequado aquele que satisfizer aos seguintes indicadores:

\begin{enumerate}[label=\roman*.]

\item regularidade: prestação dos serviços nas condições estabelecidas neste Regulamento;

\item continuidade: manutenção, em caráter permanente, da oferta dos serviços;

\item eficiência: execução dos serviços de acordo com as normas técnicas aplicáveis buscando em caráter permanente, a excelência dos serviços e assegurando, qualitativa e quantitativamente, o cumprimento dos objetivos e das metas da delegação;

\item  segurança: prestação do serviço de acordo com o estabelecido no Código de Trânsito Brasileiro, neste Regulamento e na legislação pertinente;

\item atualidade: modernidade das técnicas, dos equipamentos e das instalações, sua conservação e manutenção, bem como a melhoria e expansão do serviço na medida das necessidades dos usuários;

\item generalidade: universalidade da prestação dos serviços, isto é, serviços iguais, sem qualquer discriminação, com presteza, rapidez e segurança para todos os usuários;

\item cortesia: tratamento com urbanidade na prestação do serviço, respeito, polidez e conforto para todos os usuários; e

\item modicidade da tarifa: justa correlação entre os custos do serviço e a indenização pecuniária paga pelos usuários, expressa no valor da tarifa fixada pela SETOP.

\end{enumerate}

\begin{enumerate}[label= \S \arabic*] %parágrafo

\item Não se caracteriza como descontinuidade do serviço a sua interrupção em situação de emergência ou após prévio aviso, quando motivada por razões de segurança, por interrupção da via ou em casos fortuitos que impeçam a execução dos serviços pela Delegatária.

\item A SETOP procederá ao controle permanente da qualidade dos serviços, mediante a observância deste Regulamento e da legislação aplicável.

\end{enumerate}

\end{enumerate}

\section{Da Fiscalização}

\begin{enumerate}[resume, label=Art. \arabic*]

\item A fiscalização dos serviços de que trata este Regulamento será exercida pelo Departamento de Estradas de Rodagem do Estado de Minas Gerais – DER-MG, por intermédio de seus agentes fiscais para o desempenho desta atividade.

Parágrafo único. O DER-MG poderá celebrar convênios com outros órgãos públicos e entidades, respeitada a competência de cada qual, com a finalidade de coibir o transporte ilegal e clandestino no Estado de Minas Gerais.

\end{enumerate}

\section{Das Definições}

\begin{enumerate}[resume,label=Art. \arabic*]

\item Para efeito deste Regulamento, considera-se:

\begin{enumerate}[label=\roman*.]

\item agente fiscal: servidor designado pelo Diretor Geral do DER-MG para fiscalizar os Sistemas de Transporte Coletivo Intermunicipal e Metropolitano de Passageiros com poderes de autuação, conforme previsto neste Regulamento; (Vide Decreto nº 46.418, de 03/01/2014.)

\item bagagem etiquetada: volumes que acompanham o passageiro, transportados gratuitamente no bagageiro do veículo, sendo etiquetados, observados os seguintes limites: volumes por passageiro cujo somatório dos pesos não ultrapasse 25 kg (vinte e cinco quilogramas) e cujo somatório de volumes não ultrapasse 300 dm3 (trezentos decímetros cúbicos), limitada a maior dimensão de qualquer volume a 1 m3 (um metro cúbico);

\item bagagem não etiquetada: volumes, por passageiros, sob seu controle e responsabilidade, transportados no porta-embrulhos, observados os seguintes limites: volumes por passageiro cujo somatório dos pesos não ultrapasse 5 Kg (cinco quilogramas) e dimensões que se adaptem ao porta embrulho, desde que não comprometam a segurança e o conforto dos passageiros;

\item bagagem individual excedente: volume que ultrapassar os limites definidos nos incisos II e III, sujeita a frete;

\item bilhete de passagem: nota fiscal de prestação do serviço, conforme modelo gráfico aprovado pela Secretaria de Estado de Fazenda, de porte obrigatório pelo passageiro pagante, não podendo ser recolhido após o término da viagem;

\item capacidade nominal do veículo rodoviário: número de assentos disponíveis no veículo;

\item capacidade nominal do veículo urbano: número de assentos disponíveis no veículo, acrescentado do número de lugares disponíveis para passageiros em pé;

\item classificação das rodovias quanto à superfície de rolamento:

a) rodovia em piso I: rodovia ou via pavimentada;

b) rodovia em piso II: rodovia ou via em revestimento primário; e

c) rodovia em piso III: rodovia ou via em leito natural.

\item coeficiente de aproveitamento econômico de uma linha: relação existente, em determinado período, entre a receita apurada e a receita prevista para a linha;

\item coeficiente tarifário custo operacional a ser pago pelo passageiro para percorrer cada quilômetro de viagem;

\item delegatária: titular de delegação outorgada pela SETOP para prestar serviço nos sistemas de transporte coletivo rodoviário intermunicipal e metropolitano de passageiros no Estado;

\item encomenda: volume despachado, a critério da Delegatária, sujeito a frete, com dimensões compatíveis com a capacidade remanescente do bagageiro;

\item ficha de registro de veículo: documento emitido pela SETOP, de porte obrigatório, para operação de cada veículo nos Sistemas Intermunicipal ou Metropolitano de Passageiros;

\item frota: número de veículos registrados na SETOP por uma Delegatária do Sistema Intermunicipal de Passageiros ou do Sistema Metropolitano de Passageiros;

\item frota reserva: número de veículos disponíveis para substituir aqueles em operação, quando necessário;

\item idade média da frota: média ponderada das idades dos veículos da frota, da Delegatária, do Sistema Intermunicipal de Passageiros ou do Sistema Metropolitano de Passageiros;

\item idade do veículo: diferença entre o ano em curso e o ano do modelo da carroceria do veículo no primeiro encarroçamento, ou de fabricação do chassis no caso de veículo reencarroçado;

\item itinerário: trajeto definido pela SETOP para realização de viagem;

\item linha: serviço regular de transporte coletivo de passageiros, realizado entre dois pontos extremos, considerados início e fim da linha com características operacionais pré-fixadas;

\item motorista: preposto da empresa Delegatária, condutor do veículo, devidamente habilitado;

\item passageiro: usuário do serviço de transporte coletivo;

\item preço de passagem: valor estabelecido para cobrança do serviço prestado ao usuário;

\item quadro demonstrativo de movimento de passageiros – QDMP: documento preenchido pela Delegatária e apresentado mensalmente à SETOP, contendo informações referentes à movimentação de passageiros de cada serviço da linha;

\item quadro de tarifas: documento expedido pela SETOP, relativo a cada linha, contendo preços de passagens;

\item seção: o trecho do itinerário da linha regular em que é autorizada a cobrança de tarifa específica;

\item serviço delegado: designação genérica do objeto especificado para delegação do serviço;

\item tarifa: custo efetivo para transporte do passageiro pagante, definido pela SETOP;

\item tarifa de embarque: valor estabelecido para cobrança da utilização da infra-estrutura disponível nos terminais rodoviários de passageiros;

\item termo de manutenção: documento em que consta a declaração de responsabilidade, pela Delegatária, da manutenção do veículo de forma a garantir as condições satisfatórias de higiene, conforto e segurança;

\item transporte coletivo de passageiros: serviço público regular e permanente de transporte coletivo intermunicipal de passageiros, delegado, controlado e coordenado pela SETOP, executado sob as condições de regularidade, continuidade, eficiência, segurança, atualidade, generalidade, cortesia na sua prestação e modicidade de tarifas, aberto ao público, mediante itinerário, seccionamentos intermediários, horários e tarifas previamente definidos pela SETOP, freqüência regular, venda individual de passagens, destinado ao transporte indistinto de pessoas, compreendendo a frota cadastrada, equipamentos, instalações e as atividades inerentes à sua execução;

\item tripulação: prepostos da Delegatária para realização da viagem;

\item veículo: unidade automotora, destinada ao transporte coletivo de passageiros, nos termos do Código de Trânsito Brasileiro – CTB;

\item veículo rodoviário: unidade automotora para transporte de passageiros, dotado de poltronas individuais numeradas e reclináveis, local destinado às bagagens e sem dispositivo controlador do número de passageiros;

\item veículo urbano: unidade automotora para transporte de passageiros dotado de poltronas, com dispositivo controlador do número de passageiros;

\item viagem de reforço: viagem realizada para atendimento às demandas eventuais ou específicas do transporte coletivo de passageiros; e

\item vida útil do veículo: idade de dezoito anos. (Inciso com redação dada pelo art. 1º do decreto nº 46.680, de 19/12/2014.)

\end{enumerate}

\item Para o Sistema Intermunicipal de Passageiros, considera-se:

\begin{enumerate}[label=\roman*.]

\item atendimento parcial – ATP: serviço destinado a cumprir parte do itinerário da linha compreendida entre dois pontos de Seção ou entre um ponto de Seção e um ponto extremo, não podendo coincidir os pontos extremos do ATP com os de linha regular existente;

\item auxiliar de viagem: preposto da Delegatária que emite bilhete de passagem, auxilia o motorista, controla o fluxo de passageiros, suas respectivas bagagens etiquetadas e outras atividades afins;

\item coeficiente de aproveitamento físico de uma linha: relação existente, em determinado período, entre o número de passageiros transportados de um veículo e o número de poltronas;

\item conexão de linhas: conjugação de horários entre duas ou mais linhas ou serviços, possuindo um ponto extremo comum, fazendo-se a venda simultânea da passagem, não podendo coincidir com serviços existentes;


\item encurtamento de linha: deslocamento de ponto extremo original da linha, a partir de um dos extremos, com redução de quilometragem, não podendo o encurtamento coincidir com os pontos extremos de outra linha ou serviço existente;

\item frota especificada: somatório da frota necessária e reserva;

\item frota necessária: número de veículos necessários para cumprimento das especificações dos serviços constantes do Regime de Funcionamento das Linhas;

\item fusão de linhas: agregação de linhas existentes e operadas por uma mesma Delegatária, cujos itinerários se complementem, ainda que se superponham, não podendo coincidir os pontos extremos com linhas ou serviços existentes;

\item linha intermunicipal: linha cujos pontos extremos se localizam em municípios distintos do Estado de Minas Gerais, mesmo que o seu itinerário transponha, sem parada ou ponto de Seção, os limites do Estado, bem como os serviços autorizados por municípios vizinhos com pontos extremos próximos à divisa, permitindo conexão com a utilização de um único veículo;

\item X – ponto extremo: local onde se inicia ou termina uma viagem, havendo pré-determinação dos horários de partida;

\item ponto de parada: local destinado ao descanso e alimentação da tripulação e passageiros, devendo possuir instalações sanitárias;

\item ponto de seção: ponto limite de trecho compreendido pela Seção, sendo destinado ao embarque e desembarque de passageiros, podendo ou não ser dotado de agência de venda de passagens;

\item prolongamento de linha: deslocamento do ponto extremo original da linha a partir de um dos extremos, com acréscimo de quilometragem, não podendo coincidir os pontos extremos do mesmo com linha ou serviço existente;

\item quadro de regime de funcionamento de linha – QRF: documento expedido pela SETOP, contendo as informações básicas relativas à operação da linha;

\item restrição de seção: proibição de venda de passagem de uma Seção para outra ou de uma Seção para pontos extremos;

\item serviço comercial: serviço que opera em itinerário preferencialmente urbanizado e apresenta intensa movimentação de passageiros ao longo do dia e do itinerário, utilizando veículo urbano;

\item serviço convencional: serviço em que é utilizado veículo rodoviário;

\item sistema de transporte coletivo intermunicipal de passageiros – sistema intermunicipal de passageiros: conjunto de linhas regulares e serviços integrantes do transporte coletivo rodoviário intermunicipal, gerenciados pela SETOP; e

\item viagem: itinerário percorrido pelo veículo em um mesmo sentido, entre os pontos de origem e destino, podendo ser:

a) direta: quando não tiver ponto de Seção intermediário;

b) seccionada: quando tiver ponto de Seção intermediário; e

c) semi-direta: quando todo ponto de Seção coincidir com ponto de parada.

\end{enumerate}

\item Para o Sistema Metropolitano de Passageiros, considera-se:

\begin{enumerate}[label=\roman*.]

\item área de captação e distribuição: área localizada nas proximidades do ponto final, cabeceira da linha ou atendimento complementar, normalmente situada no entorno de vias locais ou no início de seu itinerário, desde que não seja corredor de transporte regional ou metropolitano, cujos passageiros são atendidos somente por essa Linha ou Atendimento Complementar, ou ainda área situada no entorno de parte do itinerário da Linha ou Atendimento Complementar, onde a mesma trafega com exclusividade;

\item atendimento complementar: é o atendimento com características operacionais complementares da linha já existente, mas com a mesma funcionalidade que visa atender necessidades diferenciadas, com origem na mesma Área de Captação e Distribuição da Linha;

\item bacia de captação e distribuição: compreende um conjunto de Áreas de Captação e Distribuição adjacentes, atendidas por vias arteriais, que formam uma área maior e caracterizam toda uma região de atendimento, com vias coletoras para a utilização das Linhas ou Atendimentos Complementares pertencentes às respectivas Áreas de Captação e Distribuição formadoras da Bacia;

\item cobrador: preposto da Delegatária que controla o acesso do passageiro ao veículo, o pagamento de passagem e auxilia o motorista na operação do serviço;

\item coeficiente de aproveitamento físico de uma linha: relação existente, em determinado período, entre o número de passageiros transportados por um veículo e o número de lugares oferecidos;

\item corredor de transporte regional: corredor viário, no interior das Bacias de Captação e Distribuição, para onde convergem as Linhas ou Atendimentos Complementares de transporte coletivo, que se destinam a promover a interligação entre as Áreas de Captação e Distribuição formadoras destas Bacias com o sistema arterial de vias ou entre estas Áreas com os Pólos Geradores de demandas regionais;

\item corredor de transporte metropolitano: via arterial para onde convergem as Linhas ou Atendimentos Complementares de transporte coletivo, que se destinam a promover a interligação das Bacias de Captação e Distribuição, ou a ligação dessas com os Pólos Geradores de demandas Metropolitanas;

\item espaçamento: intervalo de tempo entre o horário da viagem que está sendo realizada e o horário da viagem imediatamente antecedente ou subseqüente, previsto no Quadro de Características Operacionais – QCO;

\item frota empenhada: é o número mínimo de veículos necessários para o cumprimento das viagens preestabelecidas no QCO para cada período distinto do dia;

\item frota especificada: somatório da frota empenhada e reserva;

\item funcionalidade de uma linha ou de um atendimento complementar: caracteriza as funções operacionais vitais de cada uma das linhas, quais sejam: Área de Captação e Distribuição atendida, Bacia de Captação e Distribuição atendida, Terminais de Passageiros com sistemas tronco-alimentadores, Pólos geradores de demandas regionais ou metropolitanas e suas interligações entre si;

\item linha intramunicipal: linha cujo itinerário não ultrapassa os limites do Município;

\item linha metropolitana: serviço regular de transporte coletivo de passageiros, com características operacionais pré-fixadas pela SETOP;

\item mapa de controle operacional – M.C.O.: documento de controle operacional das Linhas e Atendimentos Complementares Metropolitanos;

\item ordem de serviço – O.S.: documento emitido pela SETOP, que determina à empresa Delegatária o cumprimento de uma rotina operacional especificada;

\item pólo gerador de demanda regional: equipamento urbano ou área que concentra o desenvolvimento de atividades que geram e atraem uma quantidade significativa de viagens de usuários do Sistema Metropolitano, pertencentes a uma determinada Bacia de Captação e Distribuição;

\item pólo gerador de demanda metropolitana: equipamento urbano ou área que concentra o desenvolvimento de atividades que geram e atraem uma quantidade significativa de viagens de usuários do Sistema Metropolitano, pertencente a mais de uma Bacia de Captação e Distribuição;

\item ponto de controle – PC: local onde se iniciam ou terminam as viagens;

\item ponto de embarque/desembarque – PED: local estabelecido para embarque ou desembarque de passageiros ao longo do itinerário;

\item quadro de característica operacional – QCO: documento expedido pela SETOP, contendo as especificações das linhas ou atendimentos complementares com seus respectivos quadros de horários;

\item sistema de transporte coletivo metropolitano de passageiros – sistema metropolitano de passageiros: conjunto de linhas ou atendimentos complementares regulares integrantes do transporte coletivo de passageiros de Região Metropolitana, gerenciados pela SETOP;

\item via local: via destinada ao atendimento dos usuários de uma Área de Captação e Distribuição;

\item via coletora: via pertencente à Bacia de Captação e Distribuição, para onde convergem as vias locais, e se destina a promover a interligação das Áreas de Captação e Distribuição, formadoras dessa Bacia com o sistema arterial de vias, ou entre essas Áreas e os Pólos Geradores de demanda regional;

\item via arterial: via que se destina a promover a ligação entre as Bacias de Captação e Distribuição, ou entre essas e os Pólos Geradores de demanda metropolitana; e

\item viagem: itinerário percorrido pelo veículo em um mesmo sentido, entre os pontos de origem e destino, podendo ser:

a) não seccionada: quando não tiver ponto de Seção intermediário;

b) seccionada: quando tiver ponto de Seção intermediário;

c) noturna: quando a viagem for realizada no período de baixa demanda, compreendida entre os horários de 22 horas às 5 horas, podendo englobar itinerário de diversas linhas e atendimentos complementares; e

d) suplementar: quando a viagem for realizada para atendimento a demandas eventuais ou específicas do transporte coletivo de passageiros.

\end{enumerate}

\end{enumerate}

\section{Da Operação da Linha}

\subsection{Do Veículo}

\begin{enumerate}[resume, label=Art. \arabic*]

\item Todo veículo, para operar no Sistema de Transporte Intermunicipal e Metropolitano de Passageiros, dependerá de cadastro prévio, nos termos estabelecidos pela SETOP.

Parágrafo único: O veículo sob arrendamento mercantil ou financiado pela Delegatária poderá ser regularmente cadastrado.

\item Será vedado o cadastramento de veículo com idade superior a dez anos.

\begin{enumerate}[label= \S \arabic*] %parágrafo

\item O veículo cadastrado poderá ser utilizado até o término de sua Vida Útil.

\item O veículo já cadastrado poderá ser transferido a outra Delegatária, desde que observado o limite da vida útil do veículo; (Parágrafo com redação dada pelo art. 2º do Decreto nº 46.680, de 19/12/2014.)

\item É vedada a alteração da característica da carroceria e capacidade nominal dos veículos registrados, sem prévia anuência da SETOP.

\item A Delegatária deverá apresentar o lay-out de pintura externa padrão de veículo para cada tipo de serviço do Sistema Intermunicipal de Passageiros.

\item A Delegatária deverá respeitar o lay-out de pintura externa padrão de veículo urbano, definido pela SETOP, para cada tipo de serviço do Sistema Metropolitano de Passageiros.

\item A Delegatária deverá apresentar o lay-out de pintura externa padrão de veículo rodoviário para utilização no Sistema Metropolitano de Passageiros.

\item Para operação dos veículos com idade superior a quinze anos de uso, nos serviços das linhas do transporte coletivo intermunicipal e metropolitano de passageiros, deverá ser apresentado o certificado de vistoria, renovável a cada seis meses, emitido pelo Instituto Nacional de Metrologia, Qualidade e Tecnologia – Inmetro –, ou por empresas por ele credenciadas, atestando serem adequadas as condições de manutenção, conservação, segurança e preservação de suas características técnicas. (Parágrafo acrescentado pelo art. 2º do Decreto nº 46.680, de 19/12/2014.) (Parágrafo com redação dada pelo art. 1º do Decreto nº 47.568, de 19/12/2018.)

\item Quando do vencimento de sua vida útil o veículo poderá ser utilizado até o dia 30 de abril do ano subsequente. (Parágrafo acrescentado pelo art. 1º do Decreto nº 47.568, de 19/12/2018.)

\end{enumerate}

\item Todo veículo deverá portar, além dos documentos exigidos pelo Código de Trânsito Brasileiro, aqueles estabelecidos pela SETOP.

\item Dar-se-á o cancelamento do cadastro:

\begin{enumerate}[label=\roman*.]

\item de ofício, quando o veículo atingir o limite da vida útil;

\item a qualquer tempo quando for considerado, através de laudo técnico do DER-MG, impróprio ou inseguro para o serviço ou a via;

\item quando for constatada irregularidade na documentação apresentada; e

\item a pedido da Delegatária.

\end{enumerate}

\item A publicidade interna em veículo só será permitida com autorização prévia da SETOP, atendidas as suas exigências e legislação específica.

\begin{enumerate}[label= \S \arabic*] %parágrafo

\item Excetuam-se no disposto neste artigo os cartazes referentes a festas regionais, comemorações oficiais e eventos patrocinados por entidades filantrópicas, devidamente reconhecidas por lei.

\item É expressamente proibida a veiculação de propaganda política de qualquer espécie.

\item A publicidade externa nos veículos, somente é admitida na parte traseira dos veículos urbanos do Sistema Metropolitano de Passageiros, observada a legislação vigente.

\end{enumerate}

\item A empresa Delegatária deverá apresentar Termo de Manutenção de seus veículos ou apresentá-los para serem vistoriados, quando solicitado pelo DER-MG.

\item  A empresa Delegatária será obrigada a substituir de imediato qualquer veículo que tenha sido retirado de circulação pelo DER-MG pelos seguintes motivos:

\begin{enumerate}[label=\roman*.]

\item o veículo não oferecer condições de segurança;

\item o veículo não oferecer condições de conforto, funcionamento ou higiene;

\item o veículo estiver em operação com o lacre do dispositivo de controle de passageiros violado;

\item o veículo não portar a Ficha de Registro;

\item não for o veículo submetido à vistoria ou apresentado Termo de Manutenção, no prazo determinado pelo DER-MG.

\end{enumerate}

\begin{enumerate}[label= \S \arabic*] %parágrafo

\item No caso dos incisos I e V, a retirada do veículo será procedida em qualquer ponto do percurso, enquanto que nos incisos II a IV, a retirada será efetivada no ponto de controle ou de parada.

\item Quando o motorista apresentar sintomas de embriaguez, o veículo deverá ser retido até a substituição daquele.

\item O veículo retirado de circulação só será liberado após vistoria realizada pelo DER-MG.

\end{enumerate}

\item Quando fora de operação por qualquer motivo, inclusive para recolhimento à garagem, ou comparecimento à vistoria, o veículo do Sistema Metropolitano será obrigado a apresentar no letreiro, reservado ao destino da viagem, as inscrições "Garagem" ou "Vistoria".

\subsection{Da Bagagem e da Encomenda}

\end{enumerate}

\begin{enumerate}[resume, label=Art. \arabic*]

\item A bagagem, definida nos incisos II e III do art. 4º, será transportada gratuitamente e terá prioridade sobre a encomenda, que ocupará o lugar remanescente no bagageiro do veículo.

Parágrafo único. Excedida a franquia fixadas nos incisos II e III do art. 4º o passageiro pagará até cinco décimos por cento do preço da passagem pelo transporte de cada quilograma de excesso.

\item O transporte de encomenda somente poderá ser feito mediante emissão de documento fiscal apropriado, observadas as prescrições legais e regulamentares, e os limites de peso do veículo estabelecidos pelo Código de Trânsito Brasileiro.

\item A Delegatária será obrigada a fornecer comprovante da bagagem e da encomenda recebidas para transporte no bagageiro, responsabilizando-se por elas.

\item Não poderão ser transportados, como bagagem ou encomenda, produtos perigosos, bem como objetos que por sua forma ou natureza, comprometam a segurança do veículo, de seus ocupantes ou de terceiros.

\item Não poderão ser transportados animais domésticos ou silvestres, a não ser quando forem objeto de lei específica.


\item A reclamação do passageiro por dano ou extravio de bagagem etiquetada deverá ser comunicada à Delegatária ou a seu preposto ao término da viagem mediante preenchimento em formulário próprio.

\begin{enumerate}[label= \S \arabic*] %parágrafo

\item A Delegatária indenizará o proprietário de bagagem etiquetada danificada ou extraviada, no prazo de trinta dias contados a partir da data da reclamação, mediante apresentação do ticket da respectiva bagagem;

\item A indenização será calculada tendo como referência o coeficiente tarifário do Sistema Intermunicipal de Passageiros, para o serviço convencional, para rodovia de Piso I, observado o seguinte critério:

\begin{enumerate}[label=\roman*.]

\item mil vezes o coeficiente tarifário, em caso de dano; e

\item três mil vezes o coeficiente tarifário, em caso de perda definitiva.

\end{enumerate}

\item A Delegatária poderá oferecer seguro complementar para coberturas excedentes da bagagem.

\end{enumerate}

\end{enumerate}

\subsection{Do Seguro do Passageiro}

\begin{enumerate}[resume, label=Art. \arabic*]

\item Será obrigatória a celebração, pela Delegatária do Sistema de Transporte Coletivo Intermunicipal Rodoviário de Passageiros, de seguro relativo a danos pessoais causados aos passageiros.

\end{enumerate}

\subsection{Da Tarifa e da Passagem}

\begin{enumerate}[resume, label=Art. \arabic*]

\item A tarifa do serviço de transporte coletivo será estipulada pela SETOP, de forma a propiciar a justa remuneração e assegurar o equilíbrio econômico-financeiro do serviço delegado.

\begin{enumerate}[label= \S \arabic*] %parágrafo

\item Caberá à SETOP estabelecer procedimentos e sistemáticas metodológicos, bem como os critérios, condições, normas e procedimentos, necessários à fixação das tarifas.

\item Será dado conhecimento público de toda atualização tarifária e do início de sua vigência, por meio de ato do Secretário de Transportes e Obras Públicas, publicado no Órgão Oficial dos Poderes do Estado.

\end{enumerate}

\item A SETOP manterá controle atualizado sobre o valor dos componentes tarifários, ficando a empresa Delegatária obrigada a fornecer informações necessárias ao estudo e cálculo das tarifas.

\item A tarifa do serviço de transporte coletivo será definida pela SETOP, sendo vedado à empresa Delegatária cobrar preço de passagem em dissonância com o valor estabelecido.

\begin{enumerate}[label= \S \arabic*] %parágrafo

\item A SETOP poderá estabelecer redutor na tarifa decorrente de outras receitas do sistema.

\item A tarifa será revista pela SETOP sempre que forem criados, alterados ou extintos tributos ou encargos legais, ou introduzidas modificações nos coeficientes de consumo pela melhoria do itinerário, ou decorrentes de atualizações tecnológicas, bem como pelas disposições legais, de comprovada repercussão na tarifa estabelecida.

\item A tarifa será revista pela SETOP com periodicidade mínima anual, observados os critérios estabelecidos em legislação e a variação dos parâmetros que compõem a base de cálculo tarifário.

\item As tarifas serão diferenciadas em função das características técnicas das rodovias, dos veículos e dos custos específicos provenientes do atendimento ao usuário.

\item A SETOP estabelecerá tarifas mínimas correspondentes a trechos percorridos, preservando o equilíbrio econômico e financeiro dos serviços prestados.

\end{enumerate}

\item O serviço convencional, convencional executivo, leito e semi-leito, o passageiro terá direito ao reembolso integral do preço do bilhete de passagem não utilizado, se apresentado até doze horas antes do início da viagem.

\item Fica vedado à Delegatária fracionar preço de passagem e estabelecer ou cancelar Seção, sem prévia anuência da SETOP.

\item O bilhete de passagem poderá ser emitido por processo mecânico, eletrônico ou similar e deverá conter os dados e forma exigidos em regulamento pela Secretaria de Estado de Fazenda.

\item A venda de passagem ou o despacho de encomenda é de responsabilidade da Delegatária, que os efetuará diretamente ou através de terceiros credenciados.

\begin{enumerate}[label= \S \arabic*] %parágrafo

\item A critério da Delegatária, a venda de bilhete de passagem poderá ser antecipada, desde que cobrado o valor vigente na data de sua emissão.

\item Uma via do bilhete de passagem se destinará ao passageiro.

\end{enumerate}

\end{enumerate}

\subsection{Da Paralisação e da Interrupção do Serviço}

\begin{enumerate}[resume, label=Art. \arabic*]

\item A SETOP poderá autorizar a paralisação parcial ou total do serviço, quando não atendidas as premissas da programação operacional ou quando ocorrer obstrução da rodovia.

Parágrafo único. A paralisação não poderá ter duração superior a trezentos e sessenta dias, sob pena de caducidade da delegação, exceto no caso de obstrução no sistema viário.

\item  Ocorrendo a interrupção de viagem, a Delegatária ficará obrigada a providenciar transporte adequado para sua conclusão, oferecer alimentação e alojamento, sem ônus para o passageiro.

\end{enumerate}

\subsection{Do Serviço de Agência e de Terminal de Passageiros}

\begin{enumerate}[resume, label=Art. \arabic*]

\item O terminal de passageiro tem como atividade própria o embarque e desembarque de passageiros, a venda de passagens, o despacho de bagagens ou encomendas e demais serviços de apoio ao usuário do transporte.

\begin{enumerate}[label= \S \arabic*] %parágrafo

\item A Delegatária poderá manter agência própria ou credenciada para venda de bilhetes de passagem nos terminais de passageiros que constarem como ponto de Seção de linha.

\item É vedada a publicidade nos guichês de venda de passagens para serviços onde exista linha ou serviço próprio de outra Delegatária.

\end{enumerate}

\item O DER-MG autorizará a utilização de terminal de passageiro quando o projeto básico de arquitetura, de reforma e as normas de funcionamento tiverem sido por ele aprovados.

\item Faculta-se às Delegatárias ou terceiros interessados a construção, administração e exploração de terminais rodoviários e pontos de paradas, observados este Regulamento e a legislação pertinente.

\item Os terminais rodoviários e os pontos de parada deverão dispor de áreas e instalações compatíveis com o seu movimento e apresentar padrões de segurança, conforto e acessibilidade ao público usuário.

\begin{enumerate}[label= \S \arabic*] %parágrafo

\item Os terminais rodoviários e os pontos de parada poderão estar localizados em instalações das empresas Delegatárias ou de terceiros, destinados especialmente para este fim e desde que aprovados pelo DER-MG.

\item Os pontos de parada deverão preferencialmente estar dispostos ao longo dos itinerários das linhas, de forma a assegurar, no curso da viagem e nos intervalos previstos, alimentação, conforto e descanso aos passageiros e tripulação dos veículos.

\end{enumerate}

\item A localização do terminal de passageiro é de responsabilidade do município interessado.


\item As Agências terão como atividades principais a venda de bilhetes de passagem e o despacho de bagagens e encomendas.

\end{enumerate}










