\chapter{DOS RECURSOS}

\begin{enumerate}[resume, label=Art. \arabic*]

\item Será liminarmente desconhecida defesa ou recursos múltiplos contra autos de infração, em 1ª e 2ª instâncias.

\item Contra o auto de infração é assegurada à Delegatária defesa perante o Diretor de Fiscalização do DER-MG, no prazo máximo de dez dias corridos, a contar do primeiro dia útil seguinte de sua intimação.

\begin{enumerate}[label= \S \arabic*] %parágrafo

\item O prazo estabelecido no caput do presente artigo será contado a partir de:

\begin{enumerate}[label=\roman*.]

\item data do auto assinado pelo próprio infrator;

\item data do Aviso de Recebimento-AR-, quando a remessa for feita por via postal;

III –\item data efetiva do recebimento do Auto no DER-MG; ou

\item data de publicação do resumo do auto no Órgão Oficial dos Poderes do Estado.

\end{enumerate}

\item A defesa deverá estar instruída com os dados e informações necessárias ao seu julgamento.

\item Contra a decisão do Diretor de Fiscalização do DER-MG caberá recurso ao Conselho de Transporte Coletivo Intermunicipal e Metropolitano – CT da SETOP, no prazo de dez dias corridos, a contar do primeiro dia útil após a data de publicação da decisão no Órgão Oficial dos Poderes do Estado.

\end{enumerate}

\item Contra a decisão do Subsecretário de Estado de Transportes, referente a fusão, prolongamento, encurtamento, atendimento parcial, alteração de itinerário, inclusão de Seção, e conexão de linhas no sistema de transporte intermunicipal, cabe recurso ao Conselho de Transporte Coletivo Intermunicipal e Metropolitano – CT da SETOP, no prazo de dez dias corridos, a contar do primeiro dia útil após a data de publicação no. Órgão Oficial dos Poderes do Estado.

\item Contra a decisão do Subsecretário de Estado de Transportes, referente à alteração de itinerário, alteração de ponto de controle, inclusão de Seção e implantação de atendimento complementar no sistema de transporte metropolitano, cabe recurso ao Conselho de Transporte Coletivo Intermunicipal e Metropolitano – CT da SETOP, no prazo de dez dias corridos, a contar do primeiro dia útil após a data de publicação no Diário Oficial do Estado.

\item As decisões do Conselho de Transporte Coletivo Intermunicipal e Metropolitano – CT da SETOP exaurem a instância administrativa.

\end{enumerate}